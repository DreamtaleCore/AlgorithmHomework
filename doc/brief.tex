%% Paper define
\documentclass[cs4size, punct, nospace, fancyhdr, fntef]{ctexart}
%\documentclass{article}
%% Dependences
\usepackage[top=0.9cm,bottom=2.5cm,left=3cm,right=2cm,includehead,includefoot]{geometry, graphicx}
%\usepackage{geometry, graphicx}
\usepackage{color}% color package
\usepackage{times}% Times New Roman font package
\usepackage{float}
\usepackage{subfig}
\usepackage{algorithm}
\usepackage{algorithmicx}
\usepackage{algpseudocode}
\usepackage{tikz}
\usetikzlibrary{arrows,shapes,chains}
%% Start of document
\begin{document}

  \section{Project/Algorithm Name}
  
  \subsection{Introduction}
    Some introduction of this project/algorithm here.
  \subsection{Pseudo code}
    \begin{algorithm}
    \caption{Euclid's algorithm}\label{euclid}
    \begin{algorithmic}[1]
    \Procedure{Euclid}{$a,b$}\Comment{The g.c.d. of a and b}
      \State $r\gets a\bmod b$
      \If{condition ok}
        \State then do it
      \EndIf
      \While{$r\not=0$}\Comment{We have the answer if r is 0}
        \State $a\gets b$
        \State $b\gets r$
        \State $r\gets a\bmod b$
      \EndWhile\label{euclidendwhile}
      \State \textbf{return} $b$\Comment{The g.c.d. is b}
    \EndProcedure
    \end{algorithmic}
    \end{algorithm}
    
  \subsection{Flowchart}
    \begin{table}[htb]
      \centering
      \begin{tabular}{c|r|l}
        \hline
        % after \\: \hline or \cline{col1-col2} \cline{col3-col4} ...
        data1 & data2 & data3 \\
        \hline
        sex & 10 & 3 \\
        hell & 9 & 6 \\
        \hline
      \end{tabular}
      \caption{algorithm's table}\label{tabledemo1}
    \end{table}
    \begin{figure}[H]
      \centering
      \includegraphics[width=10cm]{img/flowchart.jpg}
      \caption[flowchart]{algorithm's flowchart}\label{fig:flowchart}
    \end{figure}
    \begin{figure}[H]
      \centering
      % Use tikz to draw the flowchart
      \caption{flowchart2}\label{flowchart1}
    \end{figure}
    
\end{document}
