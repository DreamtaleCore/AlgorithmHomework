%% Paper define
\documentclass[cs4size, punct, nospace, fancyhdr, fntef]{ctexart}
%\documentclass{article}
%% Dependences
\usepackage[top=0.9cm,bottom=2.5cm,left=3cm,right=2cm,includehead,includefoot]{geometry, graphicx}
%\usepackage{geometry, graphicx}
\usepackage{color}% color package
\usepackage{times}% Times New Roman font package
\usepackage{float}
\usepackage{subfig}
\usepackage{algorithm}
\usepackage{algorithmicx}
\usepackage{algpseudocode}
\usepackage{tikz}
\usetikzlibrary{arrows,shapes,chains,shapes.multipart}
%% Start of document
\begin{document}

  \section{Conquer Algorithm}

  \subsection{Introduction}
    These algorithms mainly contain with:
    \begin{itemize}
      \item divide and conquer
      \item decrease and conquer
      \item change and conquer
    \end{itemize}
  \subsection{Pseudo code}
    \begin{algorithm}
    \caption{Divide-and-Conquer algorithm}\label{dnc}
    \begin{algorithmic}[1] %show line numbers
    \Require The total problem $p$
    \Ensure The result of problem $T$
    \Function{Divide-and-Conquer}{$p$}\Comment{Input problem as p}
      \If{$|p|\leq n_0$}\Comment{If p is small enough, deal with it}
        \State \textbf{return}(Adhoc($p$))
      \EndIf
      \State Divide p into sub-problems:${p_1, p2, ..., p_k}$
      \For{$i\gets 1$ to $k$}
        \State $y_i\gets$ Divide-and-Conquer($p_i$)\Comment{Deal with $p_i$ recursively}
      \EndFor
      \State $T\gets$ Merge($y_1, y_2, ..., y_k$)\Comment{Merge sub-problems}
      \State \textbf{return} $T$
    \EndFunction
    \end{algorithmic}
    \end{algorithm}

  \subsection{Flowchart}
    \begin{table}[htb]
      \centering
      \begin{tabular}{c|r|l}
        \hline
        % after \\: \hline or \cline{col1-col2} \cline{col3-col4} ...
        data1 & data2 & data3 \\
        \hline
        sex & 10 & 3 \\
        hell & 9 & 6 \\
        \hline
      \end{tabular}
      \caption{algorithm's table}\label{tabledemo1}
    \end{table}
    \begin{figure}[H]
      \centering
      \includegraphics[width=10cm]{img/flowchart.jpg}
      \caption[flowchart]{algorithm's flowchart}\label{fig:flowchart}
    \end{figure}
    \begin{figure}[H]
      \centering
      % Use tikz to draw the flowchart
      \scriptsize
      \tikzstyle{format}=[rectangle,draw,thin,fill=white]
      \tikzstyle{test}=[diamond,aspect=2,draw,thin]
      \tikzstyle{point}=[coordinate,on grid,]
      \begin{tikzpicture}[node distance=8mm, auto, >=latex', thin, start chain=going below, every join/.style={norm},]
        \node[format](n0){A};
        \node[format,below of=n0] (n1){B};
        \node[format,below of=n1] (n2){C};
        \node[format,below of=n2] (n3){D};
        \node[point,right of=n3] (p0){};
        \node[point,left of=n3] (p1){};
        \node[format,below of=n3] (n4){E};
        \node[format,below of=n4] (n5){F};
        \node[format,right of=n5] (n6){G};
        \node[test,below of=n5] (n7){H};
        \node[point,right of=n7] (p2){};
        \node[format,below of=n7] (n8){I};

        \draw[->] (n0.south) -- (n1);
        \draw[->] (n1.south) -- (n2);
        \draw[->] (n2.south) -- (n3);
        \draw[->] (n3.south) -- (n4);
        \draw[->] (n4.south) -- (n5);
        \draw[->] (n5.south) -- (n7);
        \draw[->] (n7.south) to node {Yes} (n8);
        \draw[->] (n7.east) to node {No} (p2) |- (n6.south);
        \draw[->] (n6.north) -- (p0) |- (n3.east);
      \end{tikzpicture}
      \caption{flowchart2}\label{flowchart1}
    \end{figure}
    \begin{figure}[H]
      \centering
      \tikzstyle{decision} = [diamond, draw, fill=blue!20,
        text width=4.5em, text badly centered, node distance=3cm, inner sep=0pt]
      \tikzstyle{block} = [rectangle, draw, fill=blue!20,
        text width=5em, text centered, rounded corners, minimum height=4em]
      \tikzstyle{line} = [draw, -latex']
      \tikzstyle{cloud} = [draw, ellipse,fill=red!20, node distance=3cm,
        minimum height=2em]

      \begin{tikzpicture}[node distance = 2cm, auto]
        % Place nodes
        \node [block] (init) {initialize model};
        \node [cloud, left of=init] (expert) {expert};
        \node [cloud, right of=init] (system) {system};
        \node [block, below of=init] (identify) {identify candidate models};
        \node [block, below of=identify] (evaluate) {evaluate candidate models};
        \node [block, left of=evaluate, node distance=3cm] (update) {update     model};
        \node [decision, below of=evaluate] (decide) {is best candidate     better?};
        \node [block, below of=decide, node distance=3cm] (stop) {stop};
        % Draw edges
        \path [line] (init) -- (identify);
        \path [line] (identify) -- (evaluate);
        \path [line] (evaluate) -- (decide);
        \path [line] (decide) -| node [near start] {yes} (update);
        \path [line] (update) |- (identify);
        \path [line] (decide) -- node {no}(stop);
        \path [line,dashed] (expert) -- (init);
        \path [line,dashed] (system) -- (init);
        \path [line,dashed] (system) |- (evaluate);
      \end{tikzpicture}
      \caption{flowchart3}\label{flowchart3}
    \end{figure}
    \begin{figure}[H]
      \centering
      \tikzset{
        decision/.style = {diamond, draw, fill=blue!20,
          text width=4.5em, text badly centered, node distance=3cm, inner sep=0pt},
        block/.style = {rectangle, draw, fill=blue!20,
          text width=5em, text centered, rounded corners, minimum height=4em},
        line/.style = {draw, -latex'},
        cloud/.style = {draw, ellipse,fill=red!20, node distance=3cm,
          minimum height=2em},
        subroutine/.style = {draw,rectangle split, rectangle split horizontal,
          rectangle split parts=3,minimum height=1cm,
          rectangle split part fill={red!50, green!50, blue!20, yellow!50}},
        connector/.style = {draw,circle,node distance=3cm,fill=yellow!20},
        data/.style = {draw, trapezium,node distance=3cm,fill=olive!20}
        }
      \begin{tikzpicture}[node distance = 2cm, auto]
        % Place nodes
        \node [block] (init) {initialize model};
        \node [data, left of=init] (expert) {expert};
        \node [connector, right of=init] (system) {system};
        \node [block, below of=init] (identify) {identify candidate models};
        \node [block, below of=identify] (evaluate) {evaluate candidate models};
        \node [block, left of=evaluate, node distance=3cm] (update) {update     model};
        \node [decision, below of=evaluate] (decide) {is best candidate     better?};
        \node [block, below of=decide, node distance=3cm] (test) {test};
        \node [subroutine, below of=test, node distance=3cm] (sub) {part1\nodepart{two}part2\nodepart{three}part3};
        % Draw edges
        \path [line] (init) -- (identify);
        \path [line] (identify) -- (evaluate);
        \path [line] (evaluate) -- (decide);
        \path [line] (decide) -| node [near start] {yes} (update);
        \path [line] (update) |- (identify);
        \path [line] (decide) -- node {no}(test);
        \path [line,dashed] (test) -- (sub);
        \path [line,dashed] (expert) -- (init);
        \path [line,dashed] (system) -- (init);
        \path [line,dashed] (system) |- (evaluate);
      \end{tikzpicture}
      \caption{flowchart4}\label{flowchart4}
    \end{figure}

\end{document}
